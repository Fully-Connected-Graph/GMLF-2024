\documentclass{beamer}
\usetheme{Madrid}

\title{New Energy Building Power Prediction}
\subtitle{Team: Gradient Descent Enjoyer}
\author{Jeremias Ferrao}
\date{\today}

\usepackage{graphicx}
\usepackage{listings}
\usepackage{xcolor}

\lstset{
  language=Python,
  basicstyle=\ttfamily\small,
  keywordstyle=\color{blue},
  stringstyle=\color{orange},
  commentstyle=\color{gray},
  showstringspaces=false,
  breaklines=true,
  frame=single,
  tabsize=2,
}

\begin{document}

% Introduction Slide
\begin{frame}
  \titlepage
\end{frame}

% Data Exploration Slide
\begin{frame}{Data Exploration}
  \begin{itemize}
    \item Explored the dataset containing HVAC energy consumption data
    \item Identified key features influencing energy usage
    \item Visualized data distributions and correlations
  \end{itemize}
  
\end{frame}

% Data Engineering Section
\section{Data Engineering}

% Dropping Irrelevant Features Slide
\begin{frame}{Dropping Irrelevant Features}
  \begin{itemize}
    \item Removed features with low variance or high missing rates
    \item Excluded redundant features based on correlation analysis
    \item Reduced dimensionality to improve model performance
  \end{itemize}
\end{frame}

% Processing Cyclic Features Slide
\begin{frame}{Processing Cyclic Features}
  \begin{itemize}
    \item Time features like hour and month are cyclic
    \item Applied sine and cosine transformations
    \[
    \text{sin\_hour} = \sin\left(2\pi \times \frac{\text{hour}}{24}\right)
    \]
    \[
    \text{cos\_hour} = \cos\left(2\pi \times \frac{\text{hour}}{24}\right)
    \]
    \item Captured cyclical patterns in energy consumption
  \end{itemize}
\end{frame}

% Dealing with Corrupt Data Slide
\begin{frame}{Dealing with Corrupt Data}
  \begin{itemize}
    \item Identified missing and anomalous values
    \item Imputed missing data using mean and interpolation
    \item Removed outliers based on statistical thresholds
  \end{itemize}
\end{frame}

% Model Training Section
\section{Model Training}

% Initial Approach Slide
\begin{frame}{Initial Approach}
  \begin{itemize}
    \item Started with Linear Regression as baseline
    \item Achieved MAE of initial model (e.g., 500 units)
    \item Realized need for more complex models due to nonlinearity
  \end{itemize}
\end{frame}

% Ensemble Slide
\begin{frame}{Ensemble Methods}
  \begin{itemize}
    \item Utilized Random Forest and XGBoost regressors
    \item Ensembles help in capturing diverse model strengths
    \item Improved performance over individual models
  \end{itemize}
  
\end{frame}

% Meta Learning Slide
\begin{frame}[fragile]{Meta Learning}
  \begin{itemize}
    \item Implemented StackingRegressor with ElasticNet as meta-estimator
    \item Combined predictions from base models for final output
    \item Fine-tuned hyperparameters using Grid Search
  \end{itemize}
  \vspace{5pt}
  \textbf{Code Snippet:}
  \begin{lstlisting}
# Optimize meta-estimator
meta_estimator = ElasticNet(**best_params_meta)
stacking_regressor = StackingRegressor(
    estimators=estimators,
    final_estimator=meta_estimator,
    passthrough=True,
    n_jobs=NUM_JOBS
)
stacking_regressor.fit(X, y)
  \end{lstlisting}
\end{frame}

% Feature Importance Slide
\begin{frame}{Feature Importance}
  \begin{itemize}
    \item Extracted feature importances from ensemble models
    \item Identified top contributing features to energy consumption
  \end{itemize}
\end{frame}

\end{document}